\documentclass[11pt]{ross}
\title{Important Information}

\newcommand{\spa}{\hspace*{1cm}}
\newcommand{\spz}{\hspace*{5mm}}

\usepackage{hyperref}
\usepackage{enumerate}
\usepackage{enumitem}

\begin{document}
\maketitle
 \begin{enumerate}[label=(\arabic*),itemsep=2em,topsep=-1em]
\item Several Ross documents are posted at \url{https://rossprogram.org/arrival/usa/}.  \\
Some of them require responses.  
 \begin{enumerate}[label=(\alph*),itemsep=0.5em,topsep=0em]
\item The {\it Participation Agreement} is to be printed, signed, scanned, and emailed back to 
\href{mailto:ross@rossprogram.org}{ross@rossprogram.org}.  Please bring the signed paper copy of the Agreement and give it to a Ross staff member when you arrive. 
\item The {\it Medical Authorization Form} also needs to be completed and returned.
\end{enumerate}

\item The \textit{Program Fees} document outlines payment methods.  \\
\spa Be sure to pay those fees before \feeduedate.

\item Your family is responsible for covering costs of your health
emergencies that occur during the Ross Program.\\
\spz  \textbf{All Ross Participants must have Health Insurance \\
\spz that is valid in the USA during the six weeks of the Program.} \\[5pt]
   Please scan the front and back of your medical insurance card, and email the resulting PDF file to
\texttt{medical@rossprogram.org} with the subject line ``insurance card''
followed by the name of the student. 

If you do not already have medical insurance, we can provide information on how to purchase short-term coverage valid during those six weeks.

\item The summer mailing address for Ross participants is:
\begin{quote}
\textit{$\langle$Your Student's Name$\rangle$} \\
c/o Ross Math Program\\
Ohio Dominican University \\
1216 Sunbury Road \\
Columbus, OH 43219
\end{quote}
%NOTE. All of those lines must be included in the address.

\pagebreak
\item \textbf{Arrival.}  Participants should arrive before 5:00 \peem\
  on \startsunday\today.
\begin{description}
\item 
  If you are traveling from Asia, Europe, or from the US West Coast, please arrive on
  \AdvanceDate[-1]\today. 
   Your dorm room will be available at no extra charge.  \\[5pt]
  Note: A US Visa is required for most people who are not US citizens. \\
  A Tourist Visa (B--2) is
  sufficient for your attendance at the Ross Program.
\item[By air:] Ross staff members will meet participants arriving at the John Glenn
  International Airport (CMH).  Upon arrival, 
  go to the  ``Information Desk'' in the Baggage Claim area and look for someone
  holding a ``Ross Math''  sign.  \\[1ex]
  At the end of the session, the Program will provide transportation
  to the airport, for flights departing after 2:00 \peem\ on \finishfriday\today,\ or on
  that \linebreak Thursday morning.  \\
  Everyone must be out of the
  dormitory early on \AdvanceDate\today. 
\item[By car:] For those driving to the program, please check in
  before 5:00 \peem\ at \checkinlocation. See the \textit{Driving Directions}
  documents for further information.\\
\end{description}


\item \textbf{What to bring?}
 \begin{enumerate}[label=(\alph*),itemsep=0.5em,topsep=0em]
 \item Bring clothes suitable for hot summer weather outside, and cool
   temperatures in air-conditioned buildings.  Bring (or plan to
   purchase) personal toiletries like soap, shampoo, deodorant,
   toothbrush and toothpaste, comb, brush, nail clippers, etc.
 \item Bring a pillow.  Clean sheets, pillowcase, and towel are
   provided weekly.  However, pillows are not distributed by the dorm
   staff.  You are encouraged to bring your own pillow, or purchase a
   pillow when you arrive.
 \item A microwave oven and small refrigerator will be available in
   the dorm lobby.  Those appliances are not provided in student
   rooms.
 \item Washers and Dryers are available in the dorm at no charge.  The
   Ross Program will provide bottles of detergent for everyone to use.
 \item You may bring school supplies like paper, pencils, pens,
   notebooks.  But there will also be opportunities to purchase those
   items after you arrive.
\end{enumerate}
\item \textbf{What not to bring?}
  \begin{enumerate}[label=(\alph*),itemsep=0.5em,topsep=0em]
  \item Do not bring electronic equipment like a computer, laptop,
    tablet, gaming system, music player with speakers, etc.  Ross
    counselors will confiscate such equipment, if you bring it to the
    Ross Program.
  \item Student cell phones are permitted, but their use is
    restricted: Make telephone calls only within your dorm room.
    Please do not use your phone to play games, surf the internet, or
    look up information that undercuts the process of your
    mathematical explorations.
  \item Do not bring math books, number theory textbooks, or other references.  
    The Ross Number Theory course is self-contained, with no need for 
    outside texts.
  \item Do not bring valuable items, or large amounts of cash.  The
    campus is a relatively safe place, with little crime.  But
    outsiders can sometimes sneak into the dormitory building and look
    for open doors when no one is there.
  \item Do not plan to do other activities while at the Ross Program. All of 
    your time, effort, and focus will be on the Ross courses and problem 
    sets.  There is no opportunity for you to take an on-line course, 
    prepare yourself for the SAT, write essays for your upcoming English 
    class, or read the required history textbook.
  \item There is free time for students to play outdoor games (soccer, Ultimate, 
    basketball, etc), or to play music on a University piano or your own 
    instrument.   However, preparing for some future tests or classes uses up 
    the intellectual energy that should be spent on Ross math problems!
\end{enumerate}
\end{enumerate}

\vspace*{10mm}\hrule
The Ross Program Orientation meeting will be at 7:00 \peem\ on
\startsunday\today\ in \orientationlocation.  All participants are
required to attend the orientation.



\end{document}


