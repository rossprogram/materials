\documentclass{ross}
\title{Medical Authorization}

\usepackage[most]{tcolorbox}

\renewcommand{\baselinestretch}{2}

\newtcbox{\genblank}[1][]{
    enhanced,
    arc = 1pt,
    nobeforeafter,
    tcbox raise base,
    size=small,
    width=1in,
    height=0.4in,
    overlay unbroken and first={
        \node[anchor=north,inner sep=1pt] at (frame.south) {\scriptsize\color{gray} #1};
        }
    }
\newtcbox{\wesign}[2][]{
        enhanced,
        nobeforeafter,
        tcbox raise base,
        size=small,
        width=#1,
        overlay unbroken and first={
            \node[anchor=north,inner sep=1pt] at (frame.south) {\scriptsize\color{gray} #2};
            \node[anchor=center] at (frame.center)
            {\includegraphics[width=2in]{sign-pdf.pdf}};
            }
    }

\newcommand{\blankdate}{\genblank[date]{\phantom{XX/XX/XXXX}}\ }
\newcommand{\blankemployee}{\genblank[Employee]{\phantom{gFIRST/ MIDDLE/ LAST NAME}}\ }
\newcommand{\blankaddress}{\genblank[Address]{
    \phantom{NUMBER/ STREET / CITy/ STATE/PROVINCE/ COUNTRY }}}
\newcommand{\blankwesign}{\genblank[Ross Mathematics Foundation]{
    \Huge\phantom{X/X\hspace{3in}}}}
\newcommand{\blanktheysign}{\genblank[Employee]{
    \Huge\phantom{X/X\hspace{3in}}}}

\usepackage{fontspec}
\setmainfont{Roboto}
\setsansfont{Noto Sans JP}
\setmonofont{Roboto Mono}

\begin{document}
\maketitle

%\textsc{Instructions:} Type the information where indicated here, 
%then print this document and write in the medical information %requested.
%Then scan that paper copy as a PDF file, and email it to:
%\begin{center}
%  \ttfamily \color{blue} medical@rossprogram.org
%\end{center}

During the \the\year\ Ross Program, I, \genblank[parent/guardian name]{\phantom{\hspace{2.5in}}}, 
the parent/guardian of \genblank[student name]{\phantom{\hspace{2.5in}}} (the ``Participant''), 
can be reached at \genblank[phone number]{\phantom{\hspace{1.5in}}}.

My medical insurance information is given below:
\begin{itemize}
  \item My medical insurance is provided by: \hfill \genblank[insurance company]{\phantom{\hspace{1.5in}}}
  \item The phone number for my medical insurance provider is: \hfill \genblank[insurance phone number]{\phantom{\hspace{1.5in}}}
  \item The policy holder's name is: \hfill \genblank[policy holder]{\phantom{\hspace{1.5in}}}
  \item The policy holder's DOB is: \hfill \genblank[birthday]{\phantom{\hspace{1.5in}}}
  \item My medical insurance policy number is: \hfill \genblank[policy number]{\phantom{\hspace{1.5in}}}
  \item My group number is: \hfill \genblank[group number]{\phantom{\hspace{1.5in}}}
  \item My group name is: \hfill \genblank[group name]{\phantom{\hspace{1.5in}}}
\end{itemize}

I understand that every reasonable effort will be made to contact me
at the contact information I have provided in the event of an
emergency.  In the event that I cannot be located immediately, the
authorities of the Ross Mathematics Foundation may take such temporary
measures as they deem necessary.  I give permission to physicians selected by the Ross Mathematics Foundation to treat, hospitalize, order injection, anesthesia, or surgery for the Participant.  I give permission for the release of this health information form as well as any accompanying information or medical records to medical professionals in the event of injury or illness.

To the best of my knowledge and belief, the Participant is and has
been in normal good health and is free from all communicable or
contagious diseases. Should the Participant manifest any condition
where there appears to be reasonable grounds for suspecting the
presence of a communicable or contagious disease, I agree that a
physical examination may be performed. Also, I agree that if any such
disease is found, the Participant will comply with the regular
quarantine or isolation procedures of the camp and of the community.
I agree that the Participant will submit to surveillance testing for
COVID--19 which may include mandatory antigen or PCR tests.

Describe below any medical conditions or concerns,
dietary/seasonal/medical allergies, non-allergy dietary restrictions,
and disability accommodations of which we should be aware:

\begin{tcolorbox}[enhanced,overlay unbroken and first={
  \node[anchor=north,inner sep=1pt] at (frame.south) {\scriptsize\color{gray} medical conditions};
  }]
  \vspace{1in}
\end{tcolorbox}

I understand that certain prescription medications are considered to
be ``controlled substances'' and require dispensation by a medical
professional. I also understand that I may authorize my child to
self-administer certain other medications, and/or I may authorize the
Ross Mathematics Program counselors to administer certain other
medications.

The Participant will be bringing the following prescription medications:

\begin{tcolorbox}[enhanced,overlay unbroken and first={
  \node[anchor=north,inner sep=1pt] at (frame.south) {\scriptsize\color{gray} prescription medications};
  }]
  \vspace{1in}
\end{tcolorbox}

I understand that certain prescription medications must be stored in a special manner. Please describe any special instructions related to the storage of medications for the Participant:

\begin{tcolorbox}[enhanced,overlay unbroken and first={
  \node[anchor=north,inner sep=1pt] at (frame.south) {\scriptsize\color{gray} special instructions};
  }]
  \vspace{1in}
\end{tcolorbox}

In addition to the \emph{prescription} medications listed above, the Participant will also be bringing the following \emph{non-prescription} medications:

\begin{tcolorbox}[enhanced,overlay unbroken and first={
  \node[anchor=north,inner sep=1pt] at (frame.south) {\scriptsize\color{gray} non-prescription medications};
  }]
  \vspace{1in}
\end{tcolorbox}

I authorize my child to:
\begin{itemize}
  \item self-administer the \phantom{non-}prescription medications listed above: \hfill \genblank[yes]{\phantom{XX}} \genblank[no]{\phantom{XX}}
  \item self-administer the non-prescription medications listed above:  \hfill \genblank[yes]{\phantom{XX}} \genblank[no]{\phantom{XX}}
\end{itemize}

I authorize the Ross Math Program counselors to:
\begin{itemize}
  \item administer the \phantom{non-}prescription medications listed above:  \hfill \genblank[yes]{\phantom{XX}} \genblank[no]{\phantom{XX}}
  \item administer the non-prescription medications listed above:  \hfill \genblank[yes]{\phantom{XX}} \genblank[no]{\phantom{XX}}
\end{itemize}

In the event that my child experiences a headache, fever, nausea,
sunburn, muscle pain, or other minor ailment and has not brought with
them an appropriate medication to treat such ailment, I authorize the
Ross Mathematics Program counselors to administer the following
non-prescription medications to my child:

\begin{itemize}
  \item Acetaminophen (Tylenol): \hfill \genblank[yes]{\phantom{XX}} \genblank[no]{\phantom{XX}}
  \item Aspirin: \hfill \genblank[yes]{\phantom{XX}} \genblank[no]{\phantom{XX}}
  \item Ibuprofen (Advil):  \hfill \genblank[yes]{\phantom{XX}} \genblank[no]{\phantom{XX}} 
  \item Naproxen (Aleve):  \hfill \genblank[yes]{\phantom{XX}} \genblank[no]{\phantom{XX}} 
  \item Calcium Carbonate (Tums):  \hfill \genblank[yes]{\phantom{XX}} \genblank[no]{\phantom{XX}}
  \item Bismuth Subsalicylate (Pepto-Bismol):  \hfill \genblank[yes]{\phantom{XX}} \genblank[no]{\phantom{XX}} 
  \item Calamine Lotion:  \hfill \genblank[yes]{\phantom{XX}} \genblank[no]{\phantom{XX}}
  \item Sunscreen:  \hfill \genblank[yes]{\phantom{XX}} \genblank[no]{\phantom{XX}}
\end{itemize}

%\vspace{0.25in}
%Signed \rule{3in}{.1mm}  on this  \blank{1in}{Date}{date}\\[-5pt]
%\hspace*{1in}{\footnotesize \textcolor{gray}{Parent or legal guardian} }

\vskip 1in

\hfill Signed \genblank[Parent or legal guardian]{\phantom{\hspace{3in}}} on this \genblank[Date]{\phantom{\hspace{1in}}}

\end{document}






%%% Local Variables:
%%% mode: latex
%%% TeX-master: t
%%% End:
